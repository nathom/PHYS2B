\documentclass{article}
\usepackage{amsmath}
% \usepackage{fourier}
% \usepackage{array}
\usepackage{makecell}
\usepackage{fullpage}

\renewcommand\theadalign{bc}
% \renewcommand\theadfont{\bfseries}
\renewcommand\theadgape{\Gape[4pt]}
\renewcommand\cellgape{\Gape[4pt]}

\begin{document}

\LARGE Physics 2B Quiz 5 Cheat Sheet (Chapters 29.1-29.10, 30.1-30.7)

\hrulefill
\normalsize

\begin{center}
	\begin{tabular}{|p{8cm}|p{8cm}|}
		\hline
		\multicolumn{2}{|c|}{Magnetism}                                                     \\
		\hline

		Magnetic Fields & Magnetic Forces                                                   \\

		\begin{itemize}
			\item Bio-Savart law ($ \hat{r}$ is the distance from the charge to the point being measured):
			      \[ \vec{B} = \frac{\mu_0}{4\pi} \frac{q\vec{v} \times \hat{r}}{r^2} \]
			\item Ampere's law:
			      \[ \oint \vec{B} \cdot d \vec{s} = \mu_0 I_{through} \]
			      where I is the current through the area bounded by the integration path.

			\item Faraday's law:

			      \[ \oint \vec B \cdot d \vec s = \mathcal{E}_\text{ind} = - \frac{d \Phi_B}{dt} \]
		\end{itemize}

		                &

		\begin{itemize}
			\item Magnetic force on a moving charge:
			      \[ \vec{F} = q\vec{v} \times \vec{B} \]
			\item Magnetic force on a current-carrying wire:
			      \[ \vec{F} = I\vec{l} \times \vec{B} \]
			\item Magnetic force between two parallel curren-carrying wires:
			      \[ F_{parallel wires} = \frac{\mu_0 l I_1 I_2}{2 \pi d}\]
			\item Magnetic dipole for a single loop \[\vec{\mu} = IA\]
			\item Magnetic torque on a magnetic dipole:
			      \[ \vec{\tau} = \vec{\mu} \times \vec{B}\]
		\end{itemize} \\
		\hline

		Applications of the above equations:
		\begin{itemize}
			\item Magnetic field of a wire:
			      \[B = \frac{\mu_0}{2\pi} \frac{I}{r}\]
			\item Magnetic field of a current loop:
			      \[B_{center} = \frac{mu_0}{2}\frac{NI}{R}\]
			\item Magnetic field of a solenoid:
			      \[B = \frac{\mu_0NI}{l}\]
		\end{itemize}

		                &

		Charged-particle Motion:
		\begin{itemize}
			\item No force if $\vec{v}$ is parallel to $\vec{B}$
			\item If $\vec{v}$ is perpendicular to $\vec{B}$, it undergoes circular motion at the cyclotron frequency.
			      \[ f_{cyc} = \frac{qB}{2\pi m}, r_{cyc} = \frac{mv}{qB}\]
			\item Parallel currents attract, opposite currents repel.
		\end{itemize}

		Hall Voltage (n is the charge-carrier density):
		\[ \Delta V_H = wv_dB = \frac{IB}{tne}\]                                            \\
		\hline
	\end{tabular}
	\begin{tabular}{|p{8cm}|p{8cm}|}
		\hline

		\begin{itemize}
			\item Induced current in a circuit with slide wire begin pushed

			      \[ I = \frac{\mathcal{E}}{R} = \frac{vlB}{R} \]

			\item Power dissapated by a slide wire pushed in a magnetic field
			      \[ P_\text{dissapated} = I^2 R = \frac{r^2 l^2 B^2}{R} \]
			\item Faraday's Law: The flux through a magnetic loop
			      will change if:

			      \begin{enumerate}
				      \item The loop area changes through expansion, contraction or rotation, creating a motional emf
				      \item The magnetic field changes
				            \[\mathcal{E} = \left\lvert \frac{d\phi_m}{dt}\right\rvert = \left\lvert \vec{B} \cdot \frac{d\vec{A}}{dt} + \vec{A} \cdot \frac{d\vec{B}}{dt}\right\rvert\]
				      \item When only the magnetic field is changing\[ \oint \vec{E} \cdot d\vec{s} = A\left\lvert \frac{dB}{dt}\right\rvert\]
				      \item Magnetic flux is found by\[ \Phi_m = \vec{A}\cdot\vec{B}\]
			      \end{enumerate}
		\end{itemize}


		 &

		\begin{itemize}

			\item Lenz's law:
			      \begin{itemize}
				      \item There is an induced current in a closed, conducting loop if
				            and only if the magnetic flux through the loop is changing.
				            The direction of the induced current is such that the induced
				            magnetic field opposes the change in the flux.
				      \item A decreasing magnetic field creates an induced field in the same direction as the magnetic field, and an increasing magnetic field creates an induced field in the opposite direction of the magnetic field.
			      \end{itemize}
			\item Combing Lenz's and Faraday's Laws:
			      \begin{itemize}
				      \item  \[ \mathcal{E} = \oint \vec{E} \cdot d\vec{s} = -\frac{d\Phi_m}{dt}\]
			      \end{itemize}

			\item Misc. Solenoid formulae:
				  \begin{itemize}
					  \item Flux through area A due to magnetic field of solenoid\[ \Phi_m = \frac{\mu_0 A N I_\text{sol}}{l} \]
       \item Electric fields inside and outside of solenoid:
       \[E_{in} = \frac{r}{2}\left\lvert \frac{dB}{dt} \right\rvert \]
				  \end{itemize}
		\[E_{out} = \frac{R^2}{2r}\left\lvert \frac{dB}{dt} \right\rvert \]

		\end{itemize}

		\\
		\hline
	\end{tabular}
\end{center}
\end{document}