\documentclass{article}
\usepackage{amsmath}
% \usepackage{fourier}
% \usepackage{array}
\usepackage{makecell}
\usepackage{fullpage}

\renewcommand\theadalign{bc}
% \renewcommand\theadfont{\bfseries}
\renewcommand\theadgape{\Gape[4pt]}
\renewcommand\cellgape{\Gape[4pt]}

\begin{document}

\LARGE Physics 2B Quiz 5 Cheat Sheet (Chapters 29.1-29.10, 30.1-30.7)

\hrulefill
\normalsize

\begin{center}
	\begin{tabular}{|c|c|}
		% header of table
		\hline Equation                                             & Notes                                                    \\

		% use makecell to add line breaks within table cells
		\hline $\vec B = \frac{\mu_0}{4 \pi} \frac{q \vec v \times \hat r}{r^2}
			= \frac{\mu_0}{4 \pi} \frac{q v \sin \theta }{r^2}, \text{ right hand rule}$
		                                                            & \makecell{Magnetic field of a point charge,              \\
		where $\hat r$ goes from the charge to the point being measured}                                                       \\

		$f_\text{cyc} = \frac{qB}{2 \pi m}$                         & Cyclotron frequency                                      \\

		$r_\text{cyc} = \frac{mv}{qB}$                              & Cyclotron radius                                         \\

		$\Delta V_H = w v_d B = \frac{IB}{tne}$                     &
		\makecell{Hall voltage, difference between ends of a conductor                                                         \\
		moving through magnetic field. $w$ is width, $v_d$ is drift velocity,                                                  \\
		$n$ is charge-carrier density per $m^3$.}                                                                              \\

		$\vec F_\text{wire} = I \vec l \times
		\vec B = IlB \sin \alpha, \text{ right hand rule}$          &
		\makecell{Force on a wire in a magnetic field, with $\vec l$                                                           \\
		pointing in the direction of the current.}                                                                             \\

		$F_\text{parallel wires} = \frac{\mu_0 l I_1 I_2}{2 \pi d}$ & Both wires exert equal and opposite force on each other. \\

		\hline
	\end{tabular}
\end{center}


\begin{center}
	\begin{tabular}{|p{8cm}|p{8cm}|}
		\hline
		\multicolumn{2}{|c|}{Magnetism}                                                     \\
		\hline

		Magnetic Fields & Magnetic Forces                                                   \\

		\begin{itemize}
			\item Bio-Savart law ($ \hat{r}$ is the distance from the charge to the point being measured):
			      \[ \vec{B} = \frac{\mu_0}{4\pi} \frac{q\vec{v} \times \hat{r}}{r^2} \]
			\item Ampere's law:
			      \[ \oint \vec{B} \cdot d \vec{s} = \mu_0 I_{through} \]
			      where I is the current through the area bounded by the integration path.

		\end{itemize}

		                &

		\begin{itemize}
			\item The magnetic force on a moving charge is:
			      \[ \vec{F} = q\vec{v} \times \vec{B} \]
			\item The magnetic force on a current-carrying wire is:
			      \[ \vec{F} = I\vec{l} \times \vec{B} \]
			\item The magnetic torque on a magnetic dipole is:
			      \[ \vec{\tau} = \vec{\mu} \times \vec{B}\]
		\end{itemize} \\
		\hline

		Applications of the above equations:
		\begin{itemize}
			\item Magnetic field of a wire:
			      \[B = \frac{\mu_0}{2\pi} \frac{I}{r}\]
			\item Magnetic field of a current loop:
			      \[B_{center} = \frac{mu_0}{2}\frac{NI}{R}\]
			\item Magnetic field of a solenoid:
			      \[B = \frac{\mu_0NI}{l}\]
		\end{itemize}

		                &

		Charged-particle Motion:
		\begin{itemize}
			\item No force if $\vec{v}$ is parallel to $\vec{B}$
			\item If $\vec{v}$ is perpendicular to $\vec{B}$, it undergoes circular motion at the cyclotron frequency.
			      \[ f_{cyc} = \frac{qB}{2\pi m}, r_{cyc} = \frac{mv}{qB}\]
			\item Parallel currents attract, opposite currents repel.
		\end{itemize}

		Hall Voltage (n is the charge-carrier density):
		\[ \Delta V_H = \frac{IB}{tne}\]                                                    \\
		\hline
	\end{tabular}
\end{center}
\end{document}