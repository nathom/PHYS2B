\documentclass{article}
\usepackage{amsmath}
% \usepackage{fourier}
% \usepackage{array}
\usepackage{makecell}
\usepackage{fullpage}

\renewcommand\theadalign{bc}
% \renewcommand\theadfont{\bfseries}
\renewcommand\theadgape{\Gape[4pt]}
\renewcommand\cellgape{\Gape[4pt]}

\begin{document}
\section{Quiz 4}

\begin{tabular}{|c|c|}
	% header of table
	\hline Equation                                             & Notes                                                    \\

	% use makecell to add line breaks within table cells
	\hline $\vec B = \frac{\mu_0}{4 \pi} \frac{q \vec v \times \hat r}{r^2}
		= \frac{\mu_0}{4 \pi} \frac{q v \sin \theta }{r^2}, \text{ right hand rule}$
	                                                            & \makecell{Magnetic field of a point charge,              \\
	where $\hat r$ goes from the charge to the point being measured}                                                       \\

	$f_\text{cyc} = \frac{qB}{2 \pi m}$                         & Cyclotron frequency                                      \\

	$r_\text{cyc} = \frac{mv}{qB}$                              & Cyclotron radius                                         \\

	$\Delta V_H = w v_d B = \frac{IB}{tne}$                     &
	\makecell{Hall voltage, difference between ends of a conductor                                                         \\
	moving through magnetic field. $w$ is width, $v_d$ is drift velocity,                                                  \\
	$n$ is charge-carrier density per $m^3$.}                                                                              \\

	$\vec F_\text{wire} = I \vec l \times
	\vec B = IlB \sin \alpha, \text{ right hand rule}$          &
	\makecell{Force on a wire in a magnetic field, with $\vec l$                                                           \\
	pointing in the direction of the current.}                                                                             \\

	$F_\text{parallel wires} = \frac{\mu_0 l I_1 I_2}{2 \pi d}$ & Both wires exert equal and opposite force on each other. \\

	\hline
\end{tabular}
\end{document}
